\documentclass[
11pt, % The default document font size, options: 10pt, 11pt, 12pt
%codirector, % Uncomment to add a codirector to the title page
]{charter} 



% El títulos de la memoria, se usa en la carátula y se puede usar el cualquier lugar del documento con el comando \ttitle
\titulo{Desarrollo sobre FPGA de tecnología SAR para constelación satelital} 

% Nombre del posgrado, se usa en la carátula y se puede usar el cualquier lugar del documento con el comando \degreename
\posgrado{Carrera de Especialización en Sistemas Embebidos} 
%\posgrado{Carrera de Especialización en Internet de las Cosas} 
%\posgrado{Carrera de Especialización en Inteligencia Artificial}
%\posgrado{Maestría en Sistemas Embebidos} 
%\posgrado{Maestría en Internet de las cosas}
% IMPORTANTE: no omitir titulaciones ni tildación en los nombres, también se recomienda escribir los nombres completos (tal cual los tienen en su documento)
% Tu nombre, se puede usar el cualquier lugar del documento con el comando \authorname
\autor{Ing. Martin Paura Bersan}

% El nombre del director y co-director, se puede usar el cualquier lugar del documento con el comando \supname y \cosupname y \pertesupname y \pertecosupname
\director{Ing. Daniel Jacoby}
\pertenenciaDirector{ITBA} 
\codirector{} % para que aparezca en la portada se debe descomentar la opción codirector en los parámetros de documentclass
\pertenenciaCoDirector{}

% Nombre del cliente, quien va a aprobar los resultados del proyecto, se puede usar con el comando \clientename y \empclientename
\cliente{BID. Erwin Beccari}
\empresaCliente{Proyecto FOCUS/UNSAM}
 
\fechaINICIO{20 de junio de 2024}		%Fecha de inicio de la cursada TSSE \fechaInicioName
\fechaFINALPlan{15 de agosto de 2024} 	%Fecha de final de cursada de TSSE
\fechaFINALTrabajo{15 de mayo de 2025}	%Fecha de defensa pública del trabajo final


\usepackage[utf8]{inputenc}
\usepackage[T1]{fontenc}
\usepackage{graphicx}
\usepackage{grffile}
\usepackage{longtable}
\usepackage{wrapfig}
\usepackage{rotating}
\usepackage[normalem]{ulem}
\usepackage{amsmath}
\usepackage{textcomp}
\usepackage{amssymb}
\usepackage{capt-of}
\usepackage{hyperref}
%\usepackage[left=2.00cm, right=2.50cm, top=2.50cm, bottom=2.00cm]{geometry}
%\usepackage{fancyhdr}
%\fancyhead[RO,LE]{\thepage}
%\fancyhead[LO]{\emph{\uppercase{\leftmark}}}
%\fancyfoot{}
%\renewcommand{\headrulewidth}{1.0pt}
%\pagestyle{fancy}
%\date{}
\title{Desarrollo sobre FPGA de tecnología SAR para constelación satelital}
\hypersetup{
 pdfauthor={},
 pdftitle={IEEE-830},
 pdfkeywords={},
 pdfsubject={},
 pdfcreator={Emacs 26.2 (Org mode 9.1.9)}, 
 pdflang={English}}


\begin{document}



\maketitle

\pagebreak 

\section*{Historial de cambios}
\label{sec:registro}


\begin{table}[ht]
\label{tab:registro}
\centering
\begin{tabularx}{\linewidth}{@{}|c|X|c|@{}}
\hline
\rowcolor[HTML]{C0C0C0} 
Revisión & \multicolumn{1}{c|}{\cellcolor[HTML]{C0C0C0}Detalles de los cambios realizados} & Fecha      \\ \hline
A      & Creación del documento                  &\fechaInicioName \\ \hline
B      & Primera entraga  & {15} de {julio} de 2024 \\ \hline
%2      & Se completa hasta el punto 9 inclusive	& {19} de {marzo} de 2024 \\ \hline
%3      & Se agregan puntos 11 y 12 inclusive	& {26} de {marzo} de 2024 \\ \hline
%3.1      & Se completa hasta el punto 12 inclusive	& {31} de {marzo} de 2024 \\ \hline
%4      & Se completa el plan	                                 & {2} de {abril} de 2024 \\ \hline
%4.1    & Se completa el plan con correcciones                    & {15} de {abril} de 2024 \\ \hline

% Si hay más correcciones pasada la versión 4 también se deben especificar acá

\end{tabularx}
\end{table}

\pagebreak 


\tableofcontents

\newpage

\section{Introducción}
\label{sec:org60390fa}

El presente documento detallará todos los aspectos relacionados con las especificaciones del Master Test Plan (plan maestro de pruebas) referentes al proyecto "\ttitle ".
El objetivo del proyecto es generar un relevamiento topográfico de una zona de interés, procesando los pulsos de señales electromagnéticas reflejadas en la superficie. Según se detalla en [3].
La estructura del presente documento sigue al ejemplo descrito en el Apéndice E del libro “Testing Embedded Software” de Bart Broekman y Edwin Notenboom.

\subsection{Contenido}
\label{sec:org434c3ef}

Los contenidos del presente Master Test Plan son:
\begin{enumerate}
	 \item Asignaciones.
	 \item Bases del test.
	 \item Estrategia por nivel de prueba.
\end{enumerate}


\subsection{Definiciones, Acrónimos y Abreviaturas}
\label{sec:orgb158e36}

\begin{enumerate}
\item HW	Hardware
\item FW	Firmware
\item SW 	Software
\item FOCUS Emprendimiento y proyecto del sistema general.
\item HDL 	\textit{Hardware Description Language} 
\item N/A 	No aplica 
\item RADAR	\textit{RAdio Detection And Ranging}
\item SDR 	\textit{Software Defined Radio}
\item SAR	\textit{Synthetic Aperture Radar}
\item UART 	\textit{Universal Asynchronous Receiver Transmitter} 
\item FPGA 	\textit{Field Programmable Gate Array} 
\item AXI 	\textit{Advanced eXtensible Interface} 
\item RAM 	\textit{Random Access Memory}
\item HBC	\textit{High-Bandwidth Connectivity}
\end{enumerate}


%En esta subsección se definirán todos los términos, acrónimos y
%abreviaturas utilizadas en la ERS.


\subsection{Referencias}
\label{sec:org62711e0}

\begin{enumerate}
\item [1]Especificaciones de Requerimientos de Software de la \degreename de \authorname.
\item [2]Arquitectura de Software de la \degreename de \authorname.
\item [3]Plan de Proyecto Final de la \degreename de \authorname.
\item [4]\textit{InSAR Principles: Guidelines for SAR Interferometry Processing and Interpretation (ESA TM-19)}
\item [5]https://www.earthdata.nasa.gov/learn/backgrounders/what-is-sar
\item [6]https://www.analog.com/en/resources/evaluation-hardware-and-software/evaluation-boards-kits/adalm-pluto.html
\end{enumerate}



\section{Asignación}
\label{sec:orgc1c4017}


\subsection{Responsable}
\label{sec:orgaf51da6}

El responsable de la elaboración de este documento es el ingeniero a cargo del desarrollo del proyecto, Martin Paura Bersan.


\subsection{Contratista}
\label{sec:orga40b0ee}

La asignación es ejecutada bajo responsabilidad de Martin Paura Bersan, encargado de testing del desarrollo del proyecto.

\subsection{Alcance}
\label{sec:org5ca5790}

El alcance del test de aceptación es el sistema "\ttitle" Revisión 1.0.


\subsection{Objetivos}
\label{sec:org0ae23fe}

Los objetivos son:
\begin{enumerate}
\item Determinar si el sistema cumple con los requerimientos solicitados en [1].
\item Reportar las diferencias entre lo observado y el comportamiento deseado.
\item Generar y documentar herramientas de testing que puedan ser reutilizadas en el futuro. 
\end{enumerate}

\subsection{Precondiciones}
\label{sec:org33cfcdb}

Para poder iniciar las actividades se debe cumplir:
\begin{enumerate}
\item La documentación del sistema debe estar disponible antes del 1 de enero del 2025.
\item Se debe disponer del sistema funcionando antes del 15 de enero.
\item Los procesos de testeo deben finalizar el 15 de febrero.
\end{enumerate}

\subsection{Poscondiciones}
\label{sec:org40573d1}

TBD

\section{Bases de Testeo}
Los documentos bases para diseñar el testing son:
\begin{enumerate}
\item Especificaciones de Requerimientos de Software de la \degreename de \authorname.
\item Arquitectura de Software de la \degreename de \authorname.
\item Plan de Proyecto Final de la \degreename de \authorname.
\item El libro "\textit{Testing Embedded Software}”.
\end{enumerate}

\section{Estrategia general del test}
\label{sec:orgfd5391f}

\subsection{Características de calidad}
\label{sec:org307bb59}

Se seleccionan sólo aquellas características de calidad que tienen impacto significativo en el producto.

\begin{enumerate}
\item Funcionalidad 50\%, el porcentaje de esta característica es alto debido a que el alcance del proyecto apunta a un desarrollo de un prototipo de evaluación de concepto. El cual debe realizar cálculos de precision (FFT y IFFT, transformaciones de sistemas de coordenadas) y comunicarse con otros dispositivos (Sistema SDR y computadora del sistema central) para poder responder correctamente a los requerimientos.
\item Eficiencia 20\% Este es un punto importante porque dentro de los requerimientos del sistema se incluye la capacidad del mismo para procesar 2 imágenes en menos de 1 segundo.
\item Portabilidad 20\% puesto que aun no se ha definido el hardware definitivo del producto final, es importante tener presente que la implementación final se puede realizar en otro modelo de FPGA.
\item Confiabilidad 10\% Este es un tema importante a tener en cuenta en las próximas etapas del desarrollo por eso no tiene tanto peso pero hay que tenerla en cuenta(Recuperabilidad, Control de errores).
\end{enumerate}


\subsection{Asignación de niveles de prueba a las características de calidad}
\label{sec:org94bc543}

\begin{table}[h!]
\begin{center}
\begin{tabular}{| l | c | c | c | c | c |}
\hline
\rowcolor[HTML]{C0C0C0}
Nivel de Prueba 			&Funcionalidad 	& Eficiencia 	& Portabilidad 	& Confiabilidad \\ \hline
Hardware unit test          &      			&     			& 				&   			\\ \hline
HW/FW integration test      &     ++  		&     			& 				&   	+		\\ \hline
Model in the loop            &      			&     			& 				&   			\\ \hline
Software integration test   &     ++		&     	++		& 		+		&   	+		\\ \hline
HW/SW integration test      &     ++		&     			& 				&   			\\ \hline
System test 				&     ++			&     	+		& 				&   			\\ \hline
Acceptance test 			&     ++		&     	+		& 				&   			\\ \hline
Field test 					&       		&     	++		& 				&   	++		\\ \hline
\end{tabular}
%\caption{Asignación de niveles de prueba a las características de calidad}
\end{center}
\end{table}
 
\begin{itemize}
\item[]++: El testeo de la característica de calidad se realizará a fondo en este subsistema.
\item[]+: El testeo de la característica de calidad será cubierto en este subsistema.
\item[]Celda vacía: La característica de calidad no representa un problema en este subsistema.
\end{itemize}


\section{Estrategia por nivel de prueba}
\label{sec:org49fe900}

Por cada nivel de prueba indicado en el punto 4.2 Asignación de niveles de prueba a
las características de calidad, se evalúa la estrategia con la que se lo abordará.

%Todo aquello que restrinja las decisiones relativas al diseño de la
%aplicación: Restricciones de otros estándares, limitaciones del
%hardware, etc.


\subsection{Selección de características de calidad y determinación de la importancia relativa por nivel de prueba}
\label{sec:orgd0babc0}

Se indican a continuación, para cada nivel de prueba, las características de
calidad y la importancia relativa de cada una de ellas.

\subsubsection{Hardware unit test}
\label{sec:org31d2978}

N/A

\newpage 

\subsubsection{Hardware integration test}

\begin{table}[h!]
\begin{center}
\begin{tabular}{| c | c |}
\hline
%\multicolumn{2}{ |c| }{Coches disponibles} \\ \hline
\rowcolor[HTML]{C0C0C0}
Característica de Calidad	& 	Importancia Relativa  \\ \hline
Funcionalidad          		&   80					  \\ \hline
Confiabilidad   			&   20					  \\ \hline
\end{tabular}
%\caption{Integración de hardware/firmware}
\label{tab:HW-FWIntegr}
\end{center}
\end{table} 

\subsubsection{Model in the loop}

N/A

\subsubsection{Software integration test}

\begin{table}[h!]
\begin{center}
\begin{tabular}{| c | c |}
\hline
%\multicolumn{2}{ |c| }{Coches disponibles} \\ \hline
\rowcolor[HTML]{C0C0C0}
Característica de Calidad	& 	Importancia Relativa  \\ \hline
Funcionalidad          		&   50					  \\ \hline
Eficiencia   				&   30					  \\ \hline
Portabilidad   				&   10					  \\ \hline
Confiabilidad   			&   10					  \\ \hline
\end{tabular}
%\caption{Integración de hardware/firmware}
\label{tab:HW-FWIntegr}
\end{center}
\end{table} 

\subsubsection{HW/SW integration test}

\begin{table}[h!]
\begin{center}
\begin{tabular}{| c | c |}
\hline
%\multicolumn{2}{ |c| }{Coches disponibles} \\ \hline
\rowcolor[HTML]{C0C0C0}
Característica de Calidad	& 	Importancia Relativa  \\ \hline
Funcionalidad          		&   100					  \\ \hline
\end{tabular}
%\caption{HW/SW integration test}
\label{tab:HW-SWIntegr}
\end{center}
\end{table} 


\subsubsection{System test}

\begin{table}[h!]
\begin{center}
\begin{tabular}{| c | c |}
\hline
%\multicolumn{2}{ |c| }{Coches disponibles} \\ \hline
\rowcolor[HTML]{C0C0C0}
Característica de Calidad	& 	Importancia Relativa  \\ \hline
Funcionalidad          		&   80					  \\ \hline
Eficiencia   				&   20					  \\ \hline
\end{tabular}
%\caption{System test}
\label{tab:System test}
\end{center}
\end{table} 

\subsubsection{Acceptance test}


\begin{table}[h!]
\begin{center}
\begin{tabular}{| c | c |}
\hline
%\multicolumn{2}{ |c| }{Coches disponibles} \\ \hline
\rowcolor[HTML]{C0C0C0}
Característica de Calidad	& 	Importancia Relativa  \\ \hline
Funcionalidad          		&   80					  \\ \hline
Eficiencia   				&   20					  \\ \hline
\end{tabular}
%\caption{System test}
\label{tab:System test}
\end{center}
\end{table} 

\newpage

\subsubsection{Field test}

\begin{table}[h!]
\begin{center}
\begin{tabular}{| c | c |}
\hline
%\multicolumn{2}{ |c| }{Coches disponibles} \\ \hline
\rowcolor[HTML]{C0C0C0}
Característica de Calidad	& 	Importancia Relativa  \\ \hline
Confiabilidad          		&   50					  \\ \hline
Eficiencia   				&   50					  \\ \hline
\end{tabular}
%\caption{Integración de hardware/firmware}
\label{tab:Field test}
\end{center}
\end{table} 

 

\subsection{División del sistema en subsistemas}
Basados en el funcionamiento deseado del sistema, se puede dividir en 5 partes

\begin{enumerate}
\item Parte A: Comunicación con SDR.
\item Parte B: Procesamiento SAR FDBP.
\item Parte C: Cálculo FFT e IFFT.
\item Parte D: Comunicación con computadora sistema central.
\end{enumerate}

\subsection{Determinación de la importancia realtiva de los subsistemas}

\begin{table}[h!]
\begin{center}
\begin{tabular}{| l | c |}
\hline
\rowcolor[HTML]{C0C0C0} 
Subsistema & Importancia Relativa\\\hline
Parte A: Comunicación con SDR. & 30\% \\ \hline
Parte B: Procesamiento SAR FDBP.    & 30\% \\ \hline
Parte C: Cálculo FFT e IFFT.  & 20\% \\ \hline
Parte D: Comunicación con computadora sistema central. & 20\% \\ \hline
\rowcolor[HTML]{C0C0C0} 
Total		  &100\% \\ \hline
\end{tabular}
%\caption{Integración de hardware/firmware}
\label{tab:Importancia Subsistemas}
\end{center}
\end{table} 





\subsection{Determinación de la importancia de test por combinaciones de subsistemas/características de calidad}

\begin{table}[h!]
\begin{center}
\begin{tabular}{| c | c | c | c | c | c |}
\hline
\rowcolor[HTML]{C0C0C0} 
Característica de Calidad & Importancia Relativa & Parte A & Parte B & Parte C & Parte D\\\hline
Total		  &100\% &    30\% &    30\% &    20\% &    20\%  \\ \hline
Funcionalidad & 50\% & ++ & ++ & ++ & +  \\ \hline
Eficiencia    & 20\% &   & ++  & ++ &   \\ \hline
Portabilidad  & 20\% &  & + & + &   \\ \hline
Confiabilidad & 10\% &  &  &  & ++  \\ \hline
\end{tabular}
%\caption{Integración de hardware/firmware}
%\label{tab:Field test}
\end{center}
\end{table} 

\begin{itemize}
\item[]++: El testeo de la característica de calidad se realizará a fondo en este subsistema.
\item[]+: El testeo de la característica de calidad será cubierto en este subsistema.
\item[]Celda vacía: La característica de calidad no representa un problema en este subsistema.
\end{itemize}

\end{document}
